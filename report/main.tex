\documentclass[10pt]{article}
\usepackage[spanish]{babel}
\usepackage[a4paper, tmargin=0.75in, lmargin=0.80in, rmargin=0.80in,
    bmargin=1in]{geometry}
\usepackage{hyperref}
%\usepackage{multicol}
\hypersetup{
    colorlinks=true,
    linkcolor=black,
    filecolor=magenta,      
    urlcolor=blue,
    citecolor=black,
}
%\usepackage[numbers,sort&compress]{natbib} % for a numerical citation list
%\usepackage{natbib} % to cite references by surname and year
\usepackage{graphicx}
\usepackage{amsfonts}
\usepackage{amsthm}
\usepackage{amssymb}
\usepackage{lipsum}
\usepackage{amsmath}
\usepackage{tabularx}
\usepackage{pdflscape}
\usepackage{booktabs}
\usepackage{bbm}
\usepackage{listings}
\usepackage{xcolor}

%\usepackage[
%  backend=biber,
%  style=alphabetic,
%  citestyle=alphabetic,
%  doi=true,
%  url=true,
%  isbn=false,
%  eprint=false,
%  maxbibnames=99
%]{biblatex}

%\addbibresource{references.bib}

% Definimos colores estilo "terminal con fondo negro"
\definecolor{backcolour}{rgb}{0.1,0.1,0.1}
\definecolor{codegreen}{rgb}{0,0.8,0}
\definecolor{codegray}{rgb}{0.7,0.7,0.7}
\definecolor{codepurple}{rgb}{0.8,0.6,1}
\definecolor{codewhite}{rgb}{1,1,1}


\lstdefinestyle{mypython}{
    backgroundcolor=\color{backcolour},
    basicstyle=\ttfamily\scriptsize\color{codewhite},
    commentstyle=\color{codegreen},
    keywordstyle=\color{codepurple},
    stringstyle=\color{codegreen},
    numbers=left,
    numberstyle=\tiny\color{codegray},
    breaklines=true,
    breakatwhitespace=false,
    showstringspaces=false,
    tabsize=4
}

\lstset{style=mypython}


\pagestyle{empty}


%%%%%%%%%%%%%%%%%%%%%%%%%%%%%%%%%%%%%%%%%%%%%%%%%%
%%%%%%%%%%%%%%%%%%%%%%%%%%%%%%%%%%%%%%%%%%%%%%%%%%
% ENTER SOME IMPORTANT INFORMATION
%%%%%%%%%%%%%%%%%%%%%%%%%%%%%%%%%%%%%%%%%%%%%%%%%%
%%%%%%%%%%%%%%%%%%%%%%%%%%%%%%%%%%%%%%%%%%%%%%%%%%
\newcommand{\studentname}{Hazel Shamed Sánchez Chávez}
% \newcommand{\researchcentre}{Maestría en Probabilidad y Estadística}
% \newcommand{\institution}{Centro de Investigación en Matemáticas (CIMAT)}
\newcommand{\projecttitle}{Analisis de Fractura de Roca}
\newcommand{\supervisor}{Dr. Hector Gabriel Salazar Pedroza}
%%%%%%%%%%%%%%%%%%%%%%%%%%%%%%%%%%%%%%%%%%%%%%%%%%
%%%%%%%%%%%%%%%%%%%%%%%%%%%%%%%%%%%%%%%%%%%%%%%%%%

\setlength{\parindent}{1em}
\setlength{\parskip}{1em}


\begin{document}

    \begin{center}
        {\Large{\textbf{\projecttitle}}} \\
        \vspace{2mm}
        {\Large{\textbf{Reporte de Resultados}}} \\
    \end{center}

    \vspace{5mm}
    \hrule
    \vspace{1mm}
    \hrule

    \vspace{3mm}
    \begin{tabular}{ll} 
        Estudiante:           	        & {\studentname}   \\ 
        % Programa Educativo: 	        & {\researchcentre}  \\ 
        % Institución:                 & {\institution}  \\
        Profesor: 	                 & {\supervisor}  \\ 
    \end{tabular}

    \vspace{3mm}
    \hrule
    \vspace{1mm}
    \hrule

    \begin{abstract}
        % Aquí va el resumen del reporte.
    \end{abstract}

    %%%%%%%%%%%%%%%%%%%%%%%%%%%%%%%%%%%%%%%%%%%%%%%%%%%%%%%%%%%%%%%%%%%%%%%%%%%%
    %%%%%%%%%%%%%%%%%%%%%%%%%%%%%% INTRODUCCIÓN %%%%%%%%%%%%%%%%%%%%%%%%%%%%%%%%
    %%%%%%%%%%%%%%%%%%%%%%%%%%%%%%%%%%%%%%%%%%%%%%%%%%%%%%%%%%%%%%%%%%%%%%%%%%%%

    \section{Introducción}

    \section{Planteamiento de hipótesis}

    Las ecuaciones fundamentales que describen ondas generadas por un impacto fuerte y rápido en una roca y por qué producen patrones elípticos o circulares. Las divido en niveles: desde la ecuación base hasta los modelos específicos usados en geología y mecánica de fracturas.
    
    \subsection{Ecuación de onda elastodinámica}
    
    Un impacto fuerte sobre una roca genera \textbf{ondas sísmicas} que se propagan como deformaciones en el medio sólido. La ecuación general que describe esto es la \textbf{ecuación de Navier–Cauchy}:
    
    \[ \rho \frac{\partial^2 \mathbf{u}}{\partial t^2}
    =(\lambda+\mu)\nabla(\nabla\cdot \mathbf{u})+\mu\nabla^2\mathbf{u} + \mathbf{F} \]
    %Este conjunto de ecuaciones se deriva teniendo en cuenta las tracciones a través de las superficies de un elemento de volumen, que corresponden a las componentes del tensor de tensión y las fuerzas del cuerpo que son proporcionales a la masa en el elemento de volumen. Las ecuaciones de movimiento se obtienen sumando todas las fuerzas y términos de inercia para cada componente de desplazamiento:
    
    Donde,
    \[ \mathbf{u}(x,y,z,t)=\begin{pmatrix}
        u_x(x,y,z,t)\\
        u_y(x,y,z,t)\\
        u_z(x,y,z,t)\\
    \end{pmatrix} \]
    es el campo de desplazamientos; $\rho$, densidad de la roca; $\lambda$ y $\mu$, constantes de Lamé (rigidez); y $\mathbf{F}$ la fuerza impulsiva del impacto.
    
    
    Esta ecuación genera dos ondas:
    \begin{itemize}
        \item \textbf{Ondas P:} compresionales, rápidas
        \item \textbf{Ondas S:} cortantes, más lentas
    \end{itemize}
    
    Si el impacto es puntual, la solución en 2D se vuelve casi circular; si el cuerpo tiene anisotropía o estratos, aparecen patrones elípticos.

    \subsubsection{Ecuación de onda para un impacto impulsivo}
    
    Sea $\delta$ la delta de Dirac y $F_0$ la fuerza del impacto fuerte. Este impacto se modela como una fuerza de la forma
    \[ \mathbf{F}(r,t)=F_0\delta(r)\delta(t). \]
    Es decir, concentrada en un punto ($\delta(r)$) y muy breve en el tiempo ($\delta(t)$).
    
    Por lo tanto, podemos escribir la ecuación de Navier-Cauchy como
    \[ \rho~\frac{\partial^2 u}{\partial t^2} = c^2 \nabla^2 u + F_0 \delta(r)\delta(t) \]
    donde ($c$) es la velocidad de onda en la roca.
    
    \subsection{Solución aproximada: ondas concéntricas}
    
    En un medio elástico e isótropo la solución del impacto es
    \[ u(r,t) \approx \frac{F_0}{2\pi \rho c^2 r}~ H\left(t-\frac{r}{c}\right) \]
    donde $r = \sqrt{x^2+y^2}$ y $H$ es la función escalón.
    
    Esto significa que las ondas se propagan formando círculos concéntricos.

    \subsection{Por qué aparecen patrones elípticos en rocas}
    
    Si la roca no es isotrópica (muy común en lozas), las ondas no viajan igual en todas las direcciones.
    
    Esto se modela usando velocidades diferentes según dirección ($c_x \neq c_y$)
    
    Entonces la ecuación es:
    \[ \rho \frac{\partial^2 u}{\partial t^2} = c_x^2 \frac{\partial^2 u}{\partial x^2} + c_y^2 \frac{\partial^2 u}{\partial y^2} \]
    
    La solución ya no forma círculos sino elipses
    \[ \frac{x^2}{(c_x t)^2} + \frac{y^2}{(c_y t)^2} = 1 \]
    Esto podría explicar por qué aparecen patrones elípticos en las lozas (la onda se propaga más rápido en un eje que en otro).
    
    \subsection{Fracturas generadas por impacto (modelos de grietas)}
    
    Si el impacto excede la resistencia de la roca, se generan fracturas.
    La ecuación que gobierna la apertura de una grieta está dada por \textbf{Ley de Griffith} (inicio de fractura:
    \[ \sigma_c = \sqrt{\frac{2 E \gamma}{\pi a}} \]
    donde $E$ es el módulo de Young, $\gamma$ la energía superficial y $a$ la longitud inicial de defecto.
    
    Si el impacto genera una onda con $\sigma_{\text{onda}} > \sigma_c$ entonces aparece una grieta
    
    \subsection{Forma de la grieta por impacto}
    
    Si el impacto es perpendicular
    \[ \text{Grietas radiales + círculos concéntricos (conos Hertzianos)} \]
    Si hay anisotropía:
    \[ \text{Elipses concéntricas + fracturas en forma de abanico} \]
    Esto coincide con muchos patrones en rocas y lozas.
\end{document}