\documentclass[10pt]{article}
\usepackage[spanish]{babel}
\usepackage[a4paper, tmargin=0.75in, lmargin=0.80in, rmargin=0.80in,
    bmargin=1in]{geometry}
\usepackage{hyperref}
%\usepackage{multicol}
\hypersetup{
    colorlinks=true,
    linkcolor=black,
    filecolor=magenta,      
    urlcolor=blue,
    citecolor=black,
}
%\usepackage[numbers,sort&compress]{natbib} % for a numerical citation list
%\usepackage{natbib} % to cite references by surname and year
\usepackage{graphicx}
\usepackage{amsfonts}
\usepackage{amsthm}
\usepackage{amssymb}
\usepackage{lipsum}
\usepackage{amsmath}
\usepackage{tabularx}
\usepackage{pdflscape}
\usepackage{booktabs}
\usepackage{bbm}
\usepackage{listings}
\usepackage{xcolor}

%\usepackage[
%  backend=biber,
%  style=alphabetic,
%  citestyle=alphabetic,
%  doi=true,
%  url=true,
%  isbn=false,
%  eprint=false,
%  maxbibnames=99
%]{biblatex}

%\addbibresource{references.bib}

% Definimos colores estilo "terminal con fondo negro"
\definecolor{backcolour}{rgb}{0.1,0.1,0.1}
\definecolor{codegreen}{rgb}{0,0.8,0}
\definecolor{codegray}{rgb}{0.7,0.7,0.7}
\definecolor{codepurple}{rgb}{0.8,0.6,1}
\definecolor{codewhite}{rgb}{1,1,1}


\lstdefinestyle{mypython}{
    backgroundcolor=\color{backcolour},
    basicstyle=\ttfamily\scriptsize\color{codewhite},
    commentstyle=\color{codegreen},
    keywordstyle=\color{codepurple},
    stringstyle=\color{codegreen},
    numbers=left,
    numberstyle=\tiny\color{codegray},
    breaklines=true,
    breakatwhitespace=false,
    showstringspaces=false,
    tabsize=4
}

\lstset{style=mypython}


\pagestyle{empty}


%%%%%%%%%%%%%%%%%%%%%%%%%%%%%%%%%%%%%%%%%%%%%%%%%%
%%%%%%%%%%%%%%%%%%%%%%%%%%%%%%%%%%%%%%%%%%%%%%%%%%
% ENTER SOME IMPORTANT INFORMATION
%%%%%%%%%%%%%%%%%%%%%%%%%%%%%%%%%%%%%%%%%%%%%%%%%%
%%%%%%%%%%%%%%%%%%%%%%%%%%%%%%%%%%%%%%%%%%%%%%%%%%
\newcommand{\studentname}{Hazel Shamed Sánchez Chávez}
% \newcommand{\researchcentre}{Maestría en Probabilidad y Estadística}
% \newcommand{\institution}{Centro de Investigación en Matemáticas (CIMAT)}
\newcommand{\projecttitle}{Analisis de Fractura de Roca}
\newcommand{\supervisor}{Dr. Hector Gabriel Salazar Pedroza}
%%%%%%%%%%%%%%%%%%%%%%%%%%%%%%%%%%%%%%%%%%%%%%%%%%
%%%%%%%%%%%%%%%%%%%%%%%%%%%%%%%%%%%%%%%%%%%%%%%%%%

\setlength{\parindent}{0em}
\setlength{\parskip}{1em}


\begin{document}

    \begin{center}
        {\Large{\textbf{\projecttitle}}} \\
        \vspace{2mm}
        {\Large{\textbf{Reporte de Resultados}}} \\
    \end{center}

    \vspace{5mm}
    \hrule
    \vspace{1mm}
    \hrule

    \vspace{3mm}
    \begin{tabular}{ll} 
        Estudiante:           	        & {\studentname}   \\ 
        % Programa Educativo: 	        & {\researchcentre}  \\ 
        % Institución:                 & {\institution}  \\
        Profesor: 	                 & {\supervisor}  \\ 
    \end{tabular}

    \vspace{3mm}
    \hrule
    \vspace{1mm}
    \hrule

    \begin{abstract}
        % Aquí va el resumen del reporte.
    \end{abstract}

    %%%%%%%%%%%%%%%%%%%%%%%%%%%%%%%%%%%%%%%%%%%%%%%%%%%%%%%%%%%%%%%%%%%%%%%%%%%%
    %%%%%%%%%%%%%%%%%%%%%%%%%%%%%% INTRODUCCIÓN %%%%%%%%%%%%%%%%%%%%%%%%%%%%%%%%
    %%%%%%%%%%%%%%%%%%%%%%%%%%%%%%%%%%%%%%%%%%%%%%%%%%%%%%%%%%%%%%%%%%%%%%%%%%%%

    \section{Introducción}

    \section{Planteamiento de hipótesis}

    Las ecuaciones fundamentales que describen ondas generadas por un impacto fuerte y rápido en una roca y por qué producen patrones elípticos o circulares. 
    
    \subsection{Ecuación de onda elastodinámica}
    
    Un impacto fuerte sobre una roca genera ondas que se propagan como deformaciones en un medio sólido. La ecuación general que describe esto es la \textbf{ecuación de Navier–Cauchy}:
    
    \[ \rho \frac{\partial^2 \mathbf{u}}{\partial t^2}
    =(\lambda+\mu)\nabla(\nabla\cdot \mathbf{u})+\mu\nabla^2\mathbf{u} + \mathbf{F}, \]
    
    donde,
    \[ \mathbf{u}(x_1,x_2,x_3,t)=\begin{pmatrix}
        u_1(x_1,x_2,x_3,t)\\
        u_2(x_1,x_2,x_3,t)\\
        u_3(x_1,x_2,x_3,t)\\
    \end{pmatrix} \]
    es el campo de desplazamientos; $\rho$, densidad de la roca; $\lambda$ y $\mu$, constantes de Lamé (rigidez); y $\mathbf{F}$ la fuerza impulsiva del impacto.

    La ecuación anterior describe la dinámica de un medio elástico, lineal, isótropo y homogéneo. Esta se deduce combinando la \textit{Ley de Hooke generalizada para medios isótropos} y \textit{Ecuación de movimiento (Ley de Newton)}

    \begin{itemize}
        \item \textbf{\textit{Ley de Hooke generalizada para medios isótropos:}} Para un material elástico lineal isótropo, el tensor de tensiones \(\sigma_{ij}\) se relaciona con el tensor de deformaciones infinitesimales \(\epsilon_{ij}\) mediante los coeficientes de Lamé \(\lambda\) y \(\mu\) de la siguiente manera
        \[ \sigma_{ij} = \lambda \delta_{ij} \epsilon_{kk} + 2\mu \epsilon_{ij}, \]
        donde
        \[\epsilon_{ij} = \frac{1}{2} \left( \frac{\partial u_i}{\partial x_j} + \frac{\partial u_j}{\partial x_i} \right)\]
        es el tensor de deformación, \(\epsilon_{kk} = \nabla \cdot \mathbf{u}\) es la traza del tensor de deformación) y \(\delta_{ij}\) es la delta de Kronecker.

        \item \textbf{\textit{Ecuación de movimiento (Ley de Newton):}} Para un elemento de volumen, la segunda ley de Newton en forma diferencial (ecuación de Cauchy) es
        \[ \rho \frac{\partial^2 u_i}{\partial t^2} = \frac{\partial \sigma_{ij}}{\partial x_j} + F_i \]
        donde \(F_i\) son las componentes de la fuerza de cuerpo por unidad de volumen. En notación vectorial
        \[ \rho \ddot{\mathbf{u}} = \nabla \cdot \boldsymbol{\sigma} + \mathbf{F} \]
        
    \end{itemize}
    
    
    Ahora, calculamos la divergencia del tensor de tensiones
    \[ \frac{\partial \sigma_{ij}}{\partial x_j} = \lambda \frac{\partial}{\partial x_j}(\delta_{ij} \epsilon_{kk}) + 2\mu \frac{\partial \epsilon_{ij}}{\partial x_j} \]
    
    Desarrollando tenemos
    \[\lambda \frac{\partial}{\partial x_j}(\delta_{ij} \epsilon_{kk}) = \lambda \frac{\partial \epsilon_{kk}}{\partial x_i} = \lambda \frac{\partial}{\partial x_i}(\nabla \cdot \mathbf{u})\quad\text{y}\quad 2\mu \frac{\partial \epsilon_{ij}}{\partial x_j} = \mu \left( \nabla^2 u_i + \frac{\partial}{\partial x_i}(\nabla \cdot \mathbf{u}) \right).\]
    Entonces,
    \[ \frac{\partial \sigma_{ij}}{\partial x_j} = (\lambda + \mu) \frac{\partial}{\partial x_i}(\nabla \cdot \mathbf{u}) + \mu \nabla^2 u_i \]
    y en notación vectorial
    \[ \nabla \cdot \boldsymbol{\sigma} = (\lambda + \mu) \nabla (\nabla \cdot \mathbf{u}) + \mu \nabla^2 \mathbf{u} \]
    Por lo tanto,
    \[ \rho \frac{\partial^2 \mathbf{u}}{\partial t^2} = (\lambda + \mu) \nabla (\nabla \cdot \mathbf{u}) + \mu \nabla^2 \mathbf{u} + \mathbf{F} \]
    
    Podemos interpretar a \(\rho \frac{\partial^2 \mathbf{u}}{\partial t^2}\) como la fuerza de inercia por unidad de volumen; a \((\lambda + \mu) \nabla (\nabla \cdot \mathbf{u})\) como la fuerza elástica debida a compresiones/dilataciones;
    % \(\nabla \cdot \mathbf{u}\) mide el cambio de volumen local (dilatación); \(\lambda\) y \(\mu\) son los coeficientes de Lamé, que cuantifican la resistencia del material a la compresión y al corte;
    \(\mu \nabla^2 \mathbf{u}\) es la fuerza elástica debida a deformaciones de corte;
    % \(\nabla^2 \mathbf{u}\) está relacionado con la curvatura del campo de desplazamientos;
    y \(\mathbf{F}\) son fuerzas externas por unidad de volumen (gravedad, fuerzas sísmicas, etc.).
    
    Por otro lado, si \(\mathbf{F} = 0\) y se busca soluciones de onda, se obtienen dos modos independientes:
    \begin{itemize}
        \item Ondas P (Primarias/Compresionales):
        \[ \nabla \times \mathbf{u} = 0 \quad \Rightarrow \quad \rho \ddot{\mathbf{u}} = (\lambda + 2\mu) \nabla (\nabla \cdot \mathbf{u}) \]
        Velocidad: \(c_p = \sqrt{\frac{\lambda + 2\mu}{\rho}}\).  
        Estas ondas son longitudinales (vibración en la dirección de propagación).

        \item Ondas S (Secundarias/Cortantes):
        \[ \nabla \cdot \mathbf{u} = 0 \quad \Rightarrow \quad \rho \ddot{\mathbf{u}} = \mu \nabla^2 \mathbf{u} \]
        Velocidad: \(c_s = \sqrt{\frac{\mu}{\rho}}\).  
        Estas ondas son transversales (vibración perpendicular a la dirección de propagación).
    \end{itemize}

    La Ecuación de Navier-Cauchy es fundamental en la mecánica de sólidos elásticos, describiendo cómo se mueven y deforman bajo fuerzas, relacionando desplazamiento, densidad y tensión, siendo clave para la elastodinámica (movimiento) y la elastostática (equilibrio) de materiales. Es un conjunto de ecuaciones diferenciales parciales que, en su forma general para sólidos elásticos, vinculan el desplazamiento con las fuerzas y la geometría del material, utilizando módulos de elasticidad como los de Lamé, y sirve como base para entender la propagación de ondas elásticas. 

    La ecuación combina la \textit{ley de Newton (inercia)} con la \textit{ley de Hooke (elasticidad)} en un medio continuo. Los términos elásticos representan dos mecanismos de restauración: uno por cambio de volumen (\(\nabla (\nabla \cdot \mathbf{u})\)) y otro por distorsión de forma (\(\nabla^2 \mathbf{u}\)). Esto explica por qué en sólidos existen dos tipos de ondas (P y S) con velocidades diferentes, mientras que en fluidos (donde \(\mu = 0\)) solo existen ondas P.

    \subsubsection{Ecuación de onda para un impacto impulsivo}

    Por simplicidad consideremos que un impacto se da en punto $\mathbf{0}$ (el origen) en el instante $t=0$, para cada punto $\mathbf{x}=(x_1, x_2, x_3)^{T}$ consideremos $r_\mathbf{x}:=\|\mathbf{x}\|$. Sea $\delta$ la delta de Dirac y $\mathbf{F_0}$ la fuerza del impacto fuerte. La fuerza del impacto por unidad de volumen se modela de la forma
    \[ \mathbf{F}(r_\mathbf{x},t)=\mathbf{F_0}\delta(r_\mathbf{x})\delta(t). \]
    Es decir, la fuerza $\mathbf{F_0}$ concentrada en un punto ($\delta(r)$) y muy breve en el tiempo ($\delta(t)$). Podemos escribir la ecuación de Navier-Cauchy como
    \[ \rho \frac{\partial^2 \mathbf{u}}{\partial t^2} = (\lambda+\mu)\nabla(\nabla\cdot \mathbf{u}) + \mu\nabla^2\mathbf{u} + \mathbf{F_0}\delta(r_\mathbf{x})\delta(t) \]


    \subsection{Solución fundamental para un medio infinito}
    
    Para resolver este problema, es conveniente introducir el concepto de \textit{función de Green dinámica} para el operador de Navier-Cauchy.
    
    Buscamos una solución fundamental \(\mathbf{G}(\mathbf{x}, t)\) que sea un tensor de segundo orden (3×3), que represente el desplazamiento en la dirección \(i\) debido a una fuerza puntual unitaria aplicada en la dirección \(j\). Matemáticamente, \(\mathbf{G}(\mathbf{x}, t)\) satisface
    \[ \rho \frac{\partial^2 G_{ij}(\mathbf{x}, t)}{\partial t^2} = (\lambda+\mu)\frac{\partial}{\partial x_i}(\nabla\cdot \mathbf{G}_j) + \mu\nabla^2 G_{ij}(\mathbf{x}, t) + \delta_{ij}\delta(\mathbf{x})\delta(t), \]
    donde \(\mathbf{G}_j\) es la \(j\)-ésima columna de \(\mathbf{G}\), o equivalentemente en notación vectorial:
    \[ \rho \frac{\partial^2 \mathbf{G}}{\partial t^2} = (\lambda+\mu)\nabla(\nabla\cdot \mathbf{G}) + \mu\nabla^2\mathbf{G} + \mathbf{I} \delta(\mathbf{x})\delta(t), \]
    con condiciones iniciales de reposo
    \[ \mathbf{G}(\mathbf{x}, 0) = \mathbf{0} \quad\text{y}\quad \frac{\partial \mathbf{G}}{\partial t}(\mathbf{x}, 0) = \mathbf{0}. \]
    
    Una vez encontrada \(\mathbf{G}\), la solución para nuestro impacto con fuerza \(\mathbf{F}_0\) es simplemente:
    \[ \mathbf{u}(\mathbf{x}, t) = \mathbf{G}(\mathbf{x}, t) \cdot \mathbf{F}_0 = \int_{-\infty}^{\infty} \mathbf{G}(\mathbf{x}-\mathbf{x}', t-t') \cdot \mathbf{F}_0 \delta(\mathbf{x}')\delta(t') \, d\mathbf{x}' dt'. \]
    
    Aplicando la transformada de Fourier espacial \(\hat{\mathbf{G}}(\mathbf{k}, t) = \int_{\mathbb{R}^3} \mathbf{G}(\mathbf{x}, t) e^{-i\mathbf{k}\cdot\mathbf{x}} d\mathbf{x}\), la ecuación se convierte en una ecuación diferencial ordinaria en el tiempo para cada modo \(\mathbf{k}\).
    
    Más directamente, se proyecta \(\hat{\mathbf{G}}\) en componentes longitudinal (\(P\)) y transversal (\(S\)):
    \[ \hat{\mathbf{G}} = \hat{\mathbf{G}}_P + \hat{\mathbf{G}}_S, \quad \text{donde } \mathbf{k} \times \hat{\mathbf{G}}_P = \mathbf{0} \text{ y } \mathbf{k} \cdot \hat{\mathbf{G}}_S = 0. \]
    
    Esto lleva a dos ecuaciones desacopladas:
    \[
    \rho \frac{\partial^2 \hat{\mathbf{G}}_P}{\partial t^2} = -(\lambda+2\mu) k^2 \hat{\mathbf{G}}_P + \frac{\mathbf{k}\mathbf{k}}{k^2} \delta(t)\quad\text{y}\quad
    \rho \frac{\partial^2 \hat{\mathbf{G}}_S}{\partial t^2} = -\mu k^2 \hat{\mathbf{G}}_S + \left(\mathbf{I} - \frac{\mathbf{k}\mathbf{k}}{k^2}\right) \delta(t),
    \]
    donde \(k = |\mathbf{k}|\), y \(\frac{\mathbf{k}\mathbf{k}}{k^2}\) es el proyector en la dirección de \(\mathbf{k}\).
    
    Resolviendo las EDOs anteriores (osciladores armónicos forzados por un delta) y antitransformando, se obtiene la famosa \textit{solución de Stokes} para la función de Green en un medio infinito, homogéneo e isótropo
    \[ \mathbf{G}(\mathbf{x}, t) = \frac{1}{4\pi\rho} \left\{ \frac{1}{c_p^2 r} \hat{\mathbf{x}}\hat{\mathbf{x}} \ \delta\left(t - \frac{r}{c_p}\right) + \frac{1}{c_s^2 r} (\mathbf{I} - \hat{\mathbf{x}}\hat{\mathbf{x}}) \ \delta\left(t - \frac{r}{c_s}\right) + \frac{1}{r^3} \mathbf{H}\left(t - \frac{r}{c_p}, t - \frac{r}{c_s}\right) \right\} H(t), \]
    
    donde:
    \begin{itemize}
        \item \(r = |\mathbf{x}|\), \(\hat{\mathbf{x}} = \mathbf{x}/r\) es el vector unitario radial.
        \item \(\hat{\mathbf{x}}\hat{\mathbf{x}}\) es el diádico o producto tensorial (una matriz 3×3 con componentes \(\hat{x}_i \hat{x}_j\)).
        \item \(H(t)\) es la función escalón de Heaviside, que garantiza causalidad (no hay señal antes del impacto en \(t=0\)).
        \item El tensor \(\mathbf{I} - \hat{\mathbf{x}}\hat{\mathbf{x}}\) es el proyector en el plano perpendicular a \(\hat{\mathbf{x}}\).
        \item La función tensorial \(\mathbf{H}\) representa los \textit{términos de campo cercano} (\textit{near-field}), que no son impulsivos (como las deltas) sino que tienen una duración finita \((\frac{r}{c_s} - \frac{r}{c_p})\) y decaen como \(1/r^3\). Su expresión explícita es
        \[ \mathbf{H}(\tau_p, \tau_s) = \left[ 3\hat{\mathbf{x}}\hat{\mathbf{x}} - \mathbf{I} \right] \int_{\tau_s}^{\tau_p} \tau \delta(t-\tau) d\tau = \left(3\hat{\mathbf{x}}\hat{\mathbf{x}} - \mathbf{I}\right) \left[ t \left( H(t-\tau_s) - H(t-\tau_p) \right) \right]_{\text{evaluado para } \tau_p=r/c_p, \tau_s=r/c_s}. \]
        En la práctica, \(\mathbf{H}\) aporta una señal transitoria que ocurre entre la llegada de las ondas P y S.
    \end{itemize}
    Por lo tanto, dada la fuerza impulsiva \(\mathbf{F}(\mathbf{x},t) = \mathbf{F}_0 \delta(\mathbf{x})\delta(t)\) y la función de Green \(\mathbf{G}(\mathbf{x}, t)\) derivada anteriormente, la solución explícita para el campo de desplazamientos es
    \[ \boxed{
    \mathbf{u}(\mathbf{x}, t) = \mathbf{G}(\mathbf{x}, t) \cdot \mathbf{F}_0 = \frac{H(t)}{4\pi\rho} \left[ \frac{\hat{\mathbf{x}}\hat{\mathbf{x}} \cdot \mathbf{F}_0}{c_p^2 r} \ \delta\left(t - \frac{r}{c_p}\right) + \frac{(\mathbf{I} - \hat{\mathbf{x}}\hat{\mathbf{x}}) \cdot \mathbf{F}_0}{c_s^2 r} \ \delta\left(t - \frac{r}{c_s}\right) + \frac{1}{r^3} \mathbf{H}\left(t - \frac{r}{c_p}, t - \frac{r}{c_s}\right) \cdot \mathbf{F}_0 \right]} \]

    Durante las simulaciones se observo un comportamiento circular que no encaja con nuestros patrones obtenidos en rocas.

    \subsection{Solución con condiciones iniciales y de frontera}u 

    Nuestros calcos son los siguientes

    \begin{figure}[ht]
        \centering
        \includegraphics[width=0.5\linewidth]{figures/Calcos.png}
        \caption{Enter Caption}
        \label{fig:placeholder}
    \end{figure}

    Queremos ver si a través de considerar condiciones frontera (ya que las lozas son rectangulares) los patrones obtenidos se comportan siguiendo esta ecuación.
    
    Consideremos una losa rectangular de dimensiones \(L_x \times L_y \times L_z\), con un impacto en la posición \(\mathbf{x}_0 = (x_0, y_0, z_0)^T\). La ecuación gobernante es
    \[ \rho \frac{\partial^2 \mathbf{u}}{\partial t^2} = (\lambda + \mu)\nabla(\nabla\cdot \mathbf{u}) + \mu\nabla^2\mathbf{u} + \mathbf{F}_0\delta(\mathbf{x} - \mathbf{x}_0)\delta(t) \]
    con condiciones iniciales
    \[ \mathbf{u}(\mathbf{x}, 0) = \mathbf{0}, \quad \frac{\partial \mathbf{u}}{\partial t}(\mathbf{x}, 0) = \mathbf{0} \]
    y condiciones de frontera en las seis caras del dominio.
\end{document}
